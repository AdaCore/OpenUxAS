\section{Vision}
\section{Purpose/Goals}
\section{Notional block diagram}
\section{Feature list}


\section{Unmanned Systems Autonomy Services}
UxAS is a software architecture that provides a framework to construct and deploy software services that are used to enable autonomous capabilities on-board unmanned systems. Along with the services, UxAS provides a means for defining, implementing, and exchanging inter/intra service messages.  Based on many years of design, implementation, and flight testing of collaborative algorithms for UAVs, UxAS was designed, implemented, and flight tested for AFRL's ICE-T program. UxAS relies on two software systems to provide services, the Lightweight Message Construction Protocol (LMCP), and ZeroMQ. 

LMCP was constructed by researchers at AFRL to provide a `structure for common structured data and a process for serializing objects based on those types' and `a method for encapsulating objects for transmission between applications'\cite{Duquette:2010}. The ability for LMCP to use Message Data Modules (MDM) makes it possible to define custom set of messages for each target system. An example MDM is the Common Mission Automation Services Interface (CMASI) which was designed to define data relevant to mission planning and UAV autonomy.

UxAS uses ZeroMQ to implement intra-process, inter-process, vehicle to vehicle, and vehicle to ground messaging. The authors describe ZeroMQ this way:  `looks like an embeddable networking library but acts like a concurrency framework'\cite{ZeroMQBook:2013}. Inside each UxAS process ZeroMQ publish/subscribe and push/pull nodes are used to implement a data bus that is accessible to every in-process service. The services are configured to subscribe to the LMCP messages that they require and push any generated LMCP messages to the data bus. 

UxAS is implemented in C++ and has been used with various operating systems including, LINUX, Windows, OS X, and embedded LINUX. Services inherit capabilities from a parent class that implements the core functionality, i.e. messaging, configuration, and execution. UxAS has been used for applications such as implementing collaboration algorithms onboard UAVs as well as implementing the core functionality of Unmanned Ground Sensors (UGS) with communication links to the UAVs. Figure \ref{fig:UxASServices} is a diagram of the UxAS services used to implement the AIDII system. See Table \ref{tabel:UxAS_Services} for a description of the function of each of the services.

